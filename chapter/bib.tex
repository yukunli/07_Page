
\begin{thebibliography}{100}
%\baselineskip=13pt
\addtolength{\itemsep}{-1.5ex}

\bibitem{Perlmutter1999}
 Perlmutter S, Aldering G, Goldhaber G, et al. Measurements of $\Omega$ and $\Lambda$ from 42 high-redshift supernovae[J]. The Astrophysical Journal, 1999, 517(2): 565-586.

\bibitem{SN0}
 Riess A G, Filippenko A V, Challis P, et al. Observational evidence from supernovae for an accelerating universe and a cosmological constant [J]. The Astronomical Journal, 1998, 116(3): 1009-1038.

\bibitem{Spergel:2003cb}
  Spergel D N, et al. [WMAP Collaboration],First year Wilkinson microwave anisotropy probe (WMAP) observations: determination of cosmological parameters [J].
  Astrophysical Journal Supplement Series, 2003, 148(1): 175-194.

\bibitem{Bennett:2003bz}
  Bennett C L, et al. First year Wilkinson microwave anisotropy probe (WMAP) observations: preliminary maps and basic results [J]. Astrophysical Journal Supplement Series, 2003, 148(1): 1-27.

\bibitem{Tegmark:2003ud}
  Tegmark M, Strauss M A, Blanton M R, et al. Cosmological parameters from SDSS and WMAP [J]. Physical Review D, 2004, 69(10): 103501.

\bibitem{Abazajian:2004aja}
  Abazajian K, Adelman-McCarthy J K, Ag��eros M A, et al. The second data release of the sloan digital sky survey [J]. The Astronomical Journal, 2004, 128(1): 502-512.

\bibitem{Sahni:1999gb}
  Sahni V and Starobinsky A A. The case for a positive cosmological lambda-term [J]. International Journal of Modern Physics D, 2000, 9(4): 373-443.

\bibitem{Padmanabhan:2002ji}
 Padmanabhan T. Cosmological constant-the weight of the vacuum [J]. Physics Reports, 2003, 380(5): 235-320.

\bibitem{Peebles:2002gy}
 Peebles P J E, Ratra B. The cosmological constant and dark energy [J]. Reviews of Modern Physics, 2003, 75(2): 559-606.

\bibitem{Copeland:2006wr}
  Copeland E J, Sami M, Tsujikawa S. Dynamics of dark energy [J]. International Journal of Modern Physics D, 2006, 15(11): 1753-1935.

\bibitem{Sahni:2006pa}
  Sahni V, Starobinsky A. Reconstructing dark energy [J]. International Journal of Modern Physics D, 2006, 15(12): 2105-2132.

\bibitem{Frieman:2008sn}
  Friedman J, Turner M and Huterer D. Dark energy and the accelerating universe [J]. Annual Review of Astronomy and Astrophysics, 2008, 46: 385-432.

\bibitem{Li:2011sd}
  Li M, Li X D, Wang S and Wang Y. Dark energy [J]. Communications in Theoretical Physics, 2011, 56(6): 525-562.

\bibitem{Bamba:2012cp}
  Bamba K, Capozziello S, Nojiri S, et al. Dark energy cosmology: the equivalent description via different theoretical models and cosmography tests [J]. Astrophysics and Space Science, 2012, 342(1): 155-228.

\bibitem{Weinberg:2012es}
  Weinberg D H, Mortonson M J, Eisenstein D J, et al. Observational probes of cosmic acceleration [J]. Physics Reports, 2013, 530(2): 87-255.

\bibitem{Mortonson:2013zfa}
  Mortonson M J, Weinberg D H, White M. Dark energy: a short review [J]. arXiv preprint arXiv:1401.0046, 2013.

\bibitem{Sahni1}
Sahni V, Habib S. Does Inationary Particle Production Suggest $m < 1$? [J]. Physical Review Letters, 1998, 81: 1766.

\bibitem{Parker2}
Parker L, Raval A. Nonperturbative effects of vacuum energy on the recent expansion of the universe [J]. Physical Review D, 1999, 60: 063512.

\bibitem{Dvali3}
Dvali G, Gabadadze G, Porrati M. 4D gravity on a brane in 5D Minkowski space [J]. Physics Letters B, 2000, 485(1): 208-214.

\bibitem{Deffayet4}
Deffayet C, Dvali G, Gabadadze G. Accelerated universe from gravity leaking toextra dimensions [J]. Physical Review D, 2002, 65(4): 44023.

\bibitem{Nojiri5}
Nojiri S, Odintsov S D, Sasaki M. Gauss-Bonnet dark energy [J]. Physical Review D, 2005, 71: 123509.

\bibitem{Nicolis6}
Nicolis A, Rattazzi R, Trincherini E. Galileon as a local modification of gravity [J]. Physical Review D, 2009, 79: 064036.

\bibitem{Hu7}
Hu W, Sawicki I. Models of f(R) cosmic acceleration that evade solar system tests [J]. Physical Review D, 2007, 76: 064004.

\bibitem{Starobinsky8}
Starobinsky A A. Disappearing cosmological constant in f(R) gravity [J]. Journal of Experimental and Theoretical Physics Letters, 2007, 86: 157.

\bibitem{Bengochea:2008gz}
Bengochea G R, Ferraro R. Dark torsion as the cosmic speed-up [J]. Physical Review D, 2009, 79(12): 124019.

\bibitem{Linder:2010ev}
Linder E V. Einstein's Other Gravity and the Acceleration of the Universe [J]. Physical Review D, 2010, 81(12): 127301.

\bibitem{Harko10}
Harko T, Lobo F S N, Nojiri S, Odintsov SD. f(R,T) gravity [J]. Physical Review D, 2011, 84: 024020.


\bibitem{H1}
  Jimenez R, Loeb A. Constraining cosmological parameters based on relative galaxy ages [J]. The Astrophysical Journal, 2002, 573(1): 37-42.

\bibitem{Peebles2}
Peebles P J E. Statistical analysis of catalogs of extragalactic objects. I. Theory [J]. The Astrophysical Journal, 1973, 185: 413-440.

\bibitem{H2}
  Jimenez R, Verde L, Treu T, et al. Constraints on the Equation of State of Dark Energy and the Hubble Constant from Stellar Ages and the Cosmic Microwave Background [J]. Astrophysical Journal, 2003, 593(2): 622-629.

\bibitem{H4}
  Gaztanaga E, Cabr�� A, Hui L. Clustering of luminous red galaxies�CIV. Baryon acoustic peak in the line-of-sight direction and a direct measurement of $H(z)$ [J]. Monthly Notices of the Royal Astronomical Society, 2009, 399(3): 1663-1680.

\bibitem{H5}
  Stern D, Jimenez R, Verde L, et al. Cosmic chronometers: constraining the equation of state of dark energy. I: H (z) measurements [J]. Journal of Cosmology and Astroparticle Physics, 2010, 2010(02): 008.

\bibitem{H8}
  Moresco M, Cimatti A, Jimenez R, et al. Improved constraints on the expansion rate of the Universe up to $z \thicksim 1.1$ from the spectroscopic evolution of cosmic chronometers [J]. Journal of Cosmology and Astroparticle Physics, 2012, 2012(08): 006.

\bibitem{H9}
  Blake C, Brough S, Colless M, et al. The WiggleZ Dark Energy Survey: Joint measurements of the expansion and growth history at $z< 1$ [J]. Monthly Notices of the Royal Astronomical Society, 2012, 425(1): 405-414.

\bibitem{H10}
   Chuang C H, Wang Y. Modeling the anisotropic two-Point galaxy correlation function on small scales and single-probe measurements of $H (z)$, $D_{\rm A} (z)$, and $f (z)$ and $\sigma_8 (z)$ from the Sloan Digital Sky Survey DR7 luminous red galaxies [J].Monthly notices of  the royal astronomical society, 2013, 435(1): 255-262.

\bibitem{Busca:2012bu}
  Busca N G, et al. Baryon acoustic oscillations in the Lyman-$\alpha$ forest of BOSS quasars [J]. Astronomy  Astrophysics, 2013, 552: A96.

\bibitem{Anderson:2013oza}
 Anderson L, Aubourg E, Bailey S, et al. The clustering of galaxies in the SDSS-III Baryon Oscillation Spectroscopic Survey: measuring $D_{\rm A}$ and $H$ at $z= 0.57$ from the baryon acoustic peak in the Data Release 9 spectroscopic Galaxy sample [J]. Monthly Notices of the Royal Astronomical Society, 2014, 439(1),  83-101.

\bibitem{Font-Ribera:2013wce}
  Font-Ribera A, Kirkby D, Miralda-Escud�� J, et al. Quasar-Lyman �� forest cross-correlation from BOSS DR11: Baryon Acoustic Oscillations [J]. Journal of Cosmology and Astroparticle Physics, 2014, 2014(05): 027.

\bibitem{Anderson:2013zyy}
  Anderson L, Aubourg ��, Bailey S, et al. The clustering of galaxies in the SDSS-III Baryon Oscillation Spectroscopic Survey: baryon acoustic oscillations in the Data Releases 10 and 11 Galaxy samples [J]. Monthly Notices of the Royal Astronomical Society, 2014, 441(1): 24-62.

\bibitem{H13}
  Zhang C, Zhang H, Yuan S, et al. Four new observational $H(z)$ data from luminous red galaxies in the Sloan Digital Sky Survey data release seven[J]. Research in Astronomy and Astrophysics, 2014, 14(10): 1221-1233.

\bibitem{Delubac:2014aqe}
  Delubac T, Bautista J E, Rich J, et al. Baryon Acoustic Oscillations in the Lyman-$\alpha$ forest of BOSS DR11 quasars [J]. Astronomy  Astrophysics, 2015, 574: A59.

\bibitem{Moresco:2015cya}
  Moresco M. Raising the bar: new constraints on the Hubble parameter with cosmic chronometers at $z \sim 2$ [J]. Monthly Notices of the Royal Astronomical Society: Letters, 2015, 450(1): L16-L20.

\bibitem{sandage}
 Sandage A. The Change of Redshift and Apparent Luminosity of Galaxies due to the Deceleration of Selected Expanding Universes [J]. The Astrophysical Journal, 1962, 136: 319.

\bibitem{Loeb:1998bu}
  Loeb A. Direct measurement of cosmological parameters from the cosmic deceleration of extragalactic objects [J]. The Astrophysical Journal Letters, 1998, 499(2): L111-L114.

\bibitem{Balbi:2007fx}
 Balbi A, Quercellini C. The time evolution of cosmological redshift as a test of dark energy [J]. Monthly Notices of the Royal Astronomical Society, 2007, 382(4): 1623-1629.

\bibitem{Zhang:2007zga}
  Zhang H B, Zhong W H, Zhu Z H and He S. Exploring holographic dark energy model with Sandage-Leob test [J]. Physical Review D, 2007, 76: 123508.

\bibitem{Zhang:2010im}
  Zhang J, Zhang L, Zhang X. Sandage�CLoeb test for the new agegraphic and Ricci dark energy models [J]. Physics Letters B, 2010, 691(1): 11-17.

\bibitem{Martinelli:2012vq}
  Martinelli M, Pandolfi S, Martins C, et al. Probing dark energy with redshift drift [J]. Physical Review D, 2012, 86(12): 123001.

\bibitem{Zhang:2013zyn}
 Zhang M J, Liu W B. Observational constraint on the interacting dark energy models including the Sandage�CLoeb test [J]. The European Physical Journal C, 2014, 74(5): 1-12.

\bibitem{sl4}
  Geng J J, Zhang J F, Zhang X. Quantifying the impact of future Sandage-Loeb test data on dark energy constraints [J]. Journal of Cosmology and Astroparticle Physics, 2014, 2014(07): 006.

\bibitem{sl1}
  Geng J J, Zhang J F, Zhang X. Parameter estimation with Sandage-Loeb test [J]. Journal of Cosmology and Astroparticle Physics, 2014, 2014(12): 018.

\bibitem{sl5}
  Yuan S, Zhang T J. Breaking through the high redshift bottleneck of Observational Hubble parameter Data: The Sandage-Loeb signal Scheme [J]. Journal of Cosmology and Astroparticle Physics, 2015, 2015(02): 025.

\bibitem{sl3}
 Geng J J, Li Y H, Zhang J F, et al. Redshift drift exploration for interacting dark energy [J]. The European Physical Journal C, 2015, 75(8): 1-6.

\bibitem{quint}
Peebles P J E, Ratra B. Cosmology with a time-variable cosmological'constant [J]. The Astrophysical Journal, 1988, 325: L17-L20.

\bibitem{quint1}
Ratra B, Peebles P J E. Cosmological consequences of a rolling homogeneous scalarfield [J]. Physical Review D, 1988, 37(12): 3406-3427.

\bibitem{quint2}
Wetterich C. Cosmology and the fate of dilatation symmetry [J]. Nuclear Physics B, 1988, 302(4): 668-696.

\bibitem{quint3}
Zlatev I, Wang L M, Steinhardt P J. Quintessence, cosmic coincidence, and the cosmological constant [J]. Physical Review Letters, 1999, 82(5): 896-899.

\bibitem{quint4}
Caldwell R R, Dave R, Steinhardt P J. Cosmological Imprint of an Energy Component with General Equation of State [J], Physical Review Letters, 1998, 80: 1582-1585.

\bibitem{Phant1}
Caldwell R R, Kamionkowski M,Weinberg N N. Phantom energy and cosmic doomsday [J]. Physical Review Letters, 2003, 91(7): 071301.

\bibitem{Phant2}
Caldwell R R. A phantom menace? Cosmological consequences of a dark energy component with super-negative equation of state [J]. Physics Letters B, 2002, 545(1-
2): 23-29.

\bibitem{Phant3}
Carroll S M, Hoffman M, Trodden M. Can the dark energy equation-of-state parameter $w < -1$? [J]. Physical Review D, 2003, 68: 023509.

\bibitem{Phant4}
Zhang X. Can the universe fragment into many independent causal patches at turnaround in cyclic cosmology? [J]. The European Physical Journal C, 2009, 59(4):
755-759.

\bibitem{Phant5}
Zhang X. Can black holes be torn up by phantom dark energy in cyclic cosmology? [J]. The European Physical Journal C, 2009, 60(4): 661-667.

\bibitem{Phant6}
Li X D, Wang S, Huang Q G, Zhang X, Li M. Dark energy and fate of the universe [J]. Science China Physics, Mechanics and Astronomy, 2012, 55(7): 1330-1334.

\bibitem{Quintom1}
Feng B, Wang X L, Zhang X M. Dark energy constraints from the cosmic age and supernova [J]. Physics Letters B, 2005, 607(1): 35-41.

\bibitem{Quintom2}
Guo Z K, Piao Y S, Zhang X M, Zhang Y Z. Cosmological evolution of a quintom model of dark energy [J]. Physics Letters B, 2005, 608(3): 177-182.

\bibitem{Li:2004rb}
 Li M. A model of holographic dark energy [J]. Physics Letters B, 2004, 603(1): 1-5.

\bibitem{Cai:2007us}
 Cai R G. A dark energy model characterized by the age of the universe [J]. Physics Letters B, 2007, 657(4): 228-231.

\bibitem{wei2008}
Wei H, Cai R G. A new model of agegraphic dark energy [J]. Physics Letters B, 2008, 660(3): 113-117.

\bibitem{cao2009}
Gao C, Wu F, Chen X, et al. Holographic dark energy model from Ricci scalar curvature [J]. Physical Review D, 2009, 79(4): 043511.

\bibitem{Deffayet:2001pu}
Deffayet C, Dvali G, Gabadadze G. Accelerated universe from gravity leaking to extra dimensions [J]. Physical Review D, 2002, 65(4): 044023.

\bibitem{sahni:2003pu}
Sahni V, Shtanov Y. Braneworld models of dark energy [J]. Journal of Cosmology and Astroparticle Physics, 2003, 03(11): 014.

\bibitem{Chaplygin}
Kamenshchik A, Moschella U, Pasquier V. An alternative to quintessence [J]. Physics Letters B, 2001, 511(2): 265-268.

\bibitem{bento:2002pu}
Bento M C, Bertolami O, Sen A A. Generalized Chaplygin gas, accelerated expansion, and dark-energy-matter unification [J]. Physical Review D, 2002, 66(4):
043507.




\bibitem{CPL1}
  Chevallier M, Polarski D. Accelerating universes with scaling dark matter [J]. International Journal of Modern Physics D, 2001, 10(02): 213-223.

\bibitem{CPL2}
  Linder E V. Exploring the expansion history of the universe [J]. Physical Review Letters, 2003, 90(9): 091301.

\bibitem{Huang:2004mx}
  Huang Q G, Li M. Anthropic principle favours the holographic dark energy [J]. Journal of Cosmology and Astroparticle Physics, 2005, 2005(03): 001.

\bibitem{Zhang:2014ija}
  Zhang J F, Zhao M M, Cui J L, et al. Revisiting the holographic dark energy in a non-flat universe: alternative model and cosmological parameter constraints [J]. The European Physical Journal C, 2014, 74(11): 1-8.

\bibitem{Hubble1929}
Hubble E. A relation between distance and radial velocity among extra-galactic nebulae[J]. Proceedings of the National Academy of Sciences, 1929, 15(3): 168-173.

\bibitem{Hubble1931}
Hubble E, Humason M L. The velocity-distance relation among extra-galactic nebulae[J]. The Astrophysical Journal, 1931, 74: 43.

\bibitem{Lewis:2002ija}
Lewis A, Bridle S. Cosmological parameters from CMB and other data: A Monte Carlo approach [J]. Physical Review D, 2002, 66(10): 103511.

\bibitem{Conley:2011ku}
  Conley A, Guy J, Sullivan M, et al. Supernova constraints and systematic uncertainties from the first three years of the Supernova Legacy Survey [J]. The Astrophysical Journal Supplement Series, 2011, 192(1): 1.

\bibitem{Ade:2013sjv}
Ade P A R, Aghanim N, Alves M I R, et al. Planck 2013 results. I. Overview of products and scientific results[J]. Astronomy \& Astrophysics, 2014, 571: A1.


\bibitem{Wang:2013mha}
 Wang Y, Wang S. Distance priors from Planck and dark energy constraints from current data [J]. Physical Review D, 2013, 88(4): 043522.

\bibitem{Ade:2015xua}
  Ade P A R, Aghanim N, Arnaud M, et al. Planck 2015 results. XIV. Dark energy and modified gravity [J]. arXiv preprint arXiv:1502.01590, 2015.

\bibitem{Beutler:2011}
Beutler F, et al. The 6dF Galaxy Survey: Baryon Acoustic Oscillations and the Local Hubble Constant [J]. Monthly Notices of the Royal Astronomical Society 2011, 416: 3017.

\bibitem{Padmanabhan:2012}
Padmanabhan N, Xu X, Eisenstein D J, Scalzo R, Cuesta A J, Mehta K T, Kazin E. A 2 percent distance to z=0.35 by reconstructing baryon acoustic oscillations - I. Methods and application to the Sloan Digital Sky Survey [J]. Monthly Notices of the Royal Astronomical Society, 2012, 427(3): 2132.

\bibitem{Anderson}
Anderson L, et al. The clustering of galaxies in the SDSS-III Baryon Oscillation Spectroscopic Survey: Baryon Acoustic Oscillations in the Data Release 9 Spectroscopic Galaxy Sample [J]. Monthly Notices of the Royal Astronomical Society, 2013, 427(4): 3435.

\bibitem{Blake}
Blake C, et al. The WiggleZ Dark Energy Survey: Joint measurements of the expansion and growth history at $z < 1$ [J]. Monthly Notices of the Royal Astronomical
Society, 2012, 425: 405.

\bibitem{Riess:2011yx}
  Riess A G, Macri L, Casertano S, et al. A 3\% solution: determination of the Hubble constant with the Hubble Space Telescope and wide field camera 3 [J]. The Astrophysical Journal, 2011, 730(2): 119.

\bibitem{Quercellini:2010zr}
 Quercellini C, Amendola L, Balbi A, et al. Real-time cosmology [J]. Physics Reports, 2012, 521(3): 95-134.

\bibitem{Komatsu:2008hk}
 Komatsu E, et al. [WMAP Collaboration]. Five-Year Wilkinson Microwave Anisotropy Probe (WMAP) Observations: Cosmological Interpretation [J]. The Astrophysical Journal Supplement Series, 2009, 180(2): 330-376.

\bibitem{Komatsu:2010fb}
Komatsu E, Smith K M, Dunkley J, et al. Seven year Wilkinson Microwave Anisotropy Probe (WMAP) Observations: Cosmological Interpretation [J]. The Astrophysical Journal Supplement Series, 2011, 192(2): 18.

\bibitem{msl3}
 Geng J J, Li Y H, Zhang J F, et al. Redshift drift exploration for interacting dark energy [J]. The European Physical Journal C, 2015, 75(8): 1-6.

 \bibitem{Liske:2008ph}
 Liske J, Grazian A, Vanzella E, et al. Cosmic dynamics in the era of Extremely Large Telescopes [J]. Monthly notices of the royal astronomical society, 2008, 386(3): 1192-1218.

\bibitem{phantom}
Boyle L A, Caldwell R R, Kamionkowski M. Spintessence! New models for dark matter and dark energy [J]. Physics Letters B, 2002, 545(1): 17-22.

\bibitem{k}
Armendariz-Picon C, Damour T, Mukhanov V. k-Inflation [J]. Physics Letters B, 1999, 458(2): 209-218.

\bibitem{ngcg}
Zhang X, Wu F Q, Zhang J. New generalized Chaplygin gas as a scheme for unification of dark energy and dark matter [J]. Journal of Cosmology and Astroparticle Physics, 2006, 2006(01): 003.

\bibitem{sl2}
  Geng J J, Guo R Y, He D Z, et al. Redshift drift constraints on f (T) gravity [J]. Frontiers of Physics, 2015, 10(5): 1-6.

\bibitem{Chen:2011ys}
  Chen Y, Ratra B. Hubble parameter data constraints on dark energy [J]. Physics Letters B, 2011, 703(4): 406-411.

\bibitem{Farooq:2012ju}
  Farooq O, Ratra B. Constraints on dark energy from the Lyman-$\alpha$ forest baryon acoustic oscillations measurement of the redshift 2.3 Hubble parameter [J]. Physics Letters B, 2013, 723(1): 1-6.

\bibitem{MiraldaEscude:2009uz}
  Miralda-Escude J. Comment on the claimed radial BAO detection by Gaztanaga, et al [J]. arXiv preprint arXiv:0901.1219, 2009.

\bibitem{Kazin:2010nd}
  Kazin E A, Blanton M R, Scoccimarro R, et al. Regarding the line-of-sight baryonic acoustic feature in the Sloan Digital Sky Survey and Baryon Oscillation Spectroscopic Survey luminous red galaxy samples [J]. The Astrophysical Journal, 2010, 719(2): 1032-1044.

\bibitem{Cabre:2010bc}
  Cabr�� A, Gaztanaga E. Have baryonic acoustic oscillations in the galaxy distribution really been measured? [J]. Monthly Notices of the Royal Astronomical Society: Letters, 2011, 412(1): L98-L102.

\bibitem{H15}
  Chen Y, Geng C Q, Cao S, et al. Constraints on $\phi$CDM model from strong gravitational lensing and updated Hubble parameter measurements [J]. Journal of Cosmology and Astroparticle Physics, 2015, 2015(02): 010.

\bibitem{Sahni:2014ooa}
  Sahni V, Shafieloo A, Starobinsky A A. Model-independent evidence for dark energy evolution from baryon acoustic oscillations [J]. The Astrophysical Journal Letters, 2014, 793(2): L40.

\bibitem{Corasaniti:2007bg}
Corasaniti P S, Huterer D, Melchiorri A. Exploring the dark energy redshift desert with the Sandage-Loeb test [J]. Physical Review D, 2007, 75(6): 062001.

\bibitem{Maor2001}
Maor I, Brustein R, Steinhardt P J. Limitations in using luminosity distance to determine the equation of state of the universe [J]. Physical Review Letters, 2001, 86(1): 6-9.

\end{thebibliography}
